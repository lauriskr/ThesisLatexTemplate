% this file is called up by thesis.tex
% content in this file will be fed into the main document

\chapter{State of the Art} % top level followed by section, subsection


%transition between chapters, usually no more than two parragraphs
The state of the art used in the thesis highlighted the advances in the cloud computing domain and the mobile domain...


% change according to folder and file names
\ifpdf
    \graphicspath{{X/figures/PNG/}{X/figures/PDF/}{X/figures/}}
\else
    \graphicspath{{X/figures/EPS/}{X/figures/}}
\fi

% ----------------------- State of the art ------------------------
%We will try to target 20 -25 word pages for this work
%probably after changing to Latex, the amount of pages will increase but that's because the template
%is double space and half page of each chapter is lost at the beginning, among others.

%XMPP description should fit in one page(text) + a general figure of the federated architecture
%Figures does not count as text
\section{Jabber and XMPP}
Description of Jabber and XMPP protocol and its usage.


%After the description, try to investigate about how XMPP is utilized in cloud scenarios
%both from academic and commercial points of view, it is unlikely that there is a commercial tool about it, but 
%maybe there is something, you can look for solutions for monitoring smart environments or so
%try to target 2 pages max(text). One or two paragraphs describing each solution
\section{XMPP to Cloud}
%Some academic related work that can be mentioned is described below
%try to look for them in Google scholar
%From Instant Messaging to Cloud computing, a XMPP review (paper)
%XMPP for cloud computing in bioinformatics supporting discovery and invocation of asynchronous web services
%Performance Evaluation of XMPP on the Web


%arduino along with its component should be one and half page(text) + some images
%if there are configuration we will put them in the appendix
%or we can put a footnote with a link to your github account
\section{Arduino}
Introduction to Arduino.

\subsection{Arduino Mega ADK}
Overview of the Arduino Mega ADK board and its components.

\subsection{TinkerKit}
Overview of the TinkerKit module, components and sensors.

\subsection{Modules}
Overview of the Ethernet and Wireless Shields and WiFly wireless module.

%Fuzzy logic is the main contribution, therefore, you can write here up to 5 pages
\section{Fuzzy Logic}
Description of fuzzy logic and its uses.

%This one is not necessary, since we will just mention the tool and it will be cited in chapter 4
%\section{Power supply}
%Overview of the PeakTech 1890 power supply.



%add figures
%\begin{figure}
%\centering
%\includegraphics[width=0.65\textwidth]{2/figures/Cloud/funambolArchitecture.png}
%\caption{Funambol architecture}
%\label{fig:funambolArchitecture}
%\end{figure}


% ---------------------------------------------------------------------------
% ----------------------- end of thesis sub-document ------------------------
% --------------------------------------------------------------------------- 

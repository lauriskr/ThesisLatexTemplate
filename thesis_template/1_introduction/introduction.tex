
% this file is called up by thesis.tex
% content in this file will be fed into the main document

%: ----------------------- introduction file header -----------------------

\chapter{Introduction}

% the code below specifies where the figures are stored
\ifpdf
    \graphicspath{{1_introduction/figures/PNG/}{1_introduction/figures/PDF/}{1_introduction/figures/}}
\else
    \graphicspath{{1_introduction/figures/EPS/}{1_introduction/figures/}}
\fi

% ----------------------------------------------------------------------
%: ----------------------- introduction content -----------------------
% ----------------------------------------------------------------------



%: ----------------------- HELP: latex document organisation
% the commands below help you to subdivide and organise your thesis
%    \chapter{}       = level 1, top level
%    \section{}       = level 2
%    \subsection{}    = level 3
%    \subsubsection{} = level 4
% note that everything after the percentage sign is hidden from output



\section{Introduction} % section headings are printed smaller than chapter names
% intro
\todo{Write this part}
Briefly summarize the question (you will be stating the question in detail later), and perhaps give an overview of your main results. (it is not just a description of the contents of each section)


\subsection{Motivation}
The developed prototype \cite{prev_thesis} works well when the general idea and goal is considered. However, the lack of flexibility of data transmission intervals and high power needs constrain the implementation usages. In order to take full advantage of the prototype, power usage should be reduced and more flexibility added to the communication between the server-side client and sensor modules.  

\subsection{Contributions}

A more flexible data collection solution is developed based on HTTP. The changes made enable more energy-efficient data collection by predicting sensor data when possible. A fuzzy control system was developed to decide if sensor data is predictable and for what time period.  Moreover, the prototype's Arduino-based sensor module has been improved to support sleep mode while no sensor data is sent to the server-side client. 

Power consumption before and after the changes was tested to indicate changes' effect on battery life. \todo{mention peaktech power supply here?}

\subsection{Outline}

%for example
%\noindent \textbf{Chapter 2}: discusses the state of the art addressed by this thesis. 

\noindent \textbf{Chapter 2}: introduces the Arduino framework and modules used in the prototype. Finally, a description of fuzzy control systems and simple linear regression is given.
\newline

\noindent \textbf{Chapter 3}: describes the existing prototype and gives an overview of its implementation details. Later, an overview of identified areas of improvement is given.
\newline

\noindent \textbf{Chapter 4}: explains the improvements made to the prototype. Firstly, describes changes made to the Arduino sensor module. Secondly, changes in the communication between the sensor module and the server-side client are discussed. Lastly, an overview of the server-side fuzzy control system and data prediction techniques is given. The chapter is ended by a power consumption comparison between the initial and improved prototype.
\newline

\todo{add more chapters here when they're ready}

%\noindent \textbf{Chapter 3}: explains the problems regarding the combination and the invocation of cloud services from the mobile phone. 



